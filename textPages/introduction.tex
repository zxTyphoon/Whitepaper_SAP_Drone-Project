\chapter{Introduction}
\label{introduction}
\nocite{*}

\section{Background}
This document details the development of two \acp{PoC} as part of the \ac{ASTRID} project at SAP Labs India. \ac{ASTRID}, an initiative driven by the Innovations Team, 
seeks to explore the potential of \acp{LLM} in enabling seamless and intuitive voice-based control for drones. 

Modern drones offer sophisticated capabilities, yet their control interfaces often require significant expertise, limiting their accessibility for non-technical users. 
By leveraging advancements in \acp{LLM}, the \ac{ASTRID} project aims to simplify drone operation by enabling natural language voice commands. This vision has applications across domains such as surveillance, remote monitoring, and emergency response.

The project's practical implementation was tested on a DJI Mini 3 Pro drone, a compact and versatile drone known for its advanced capabilities, including high-quality video capture and stability. 
The drone connects to a dedicated controller, which is paired with a mobile device running the DJI app. This setup provides both control inputs and real-time video feedback, forming the base infrastructure for the integration of voice-based operations and low-latency streaming solutions.

The \acp{PoC} developed within this project aim to address key challenges in achieving this vision, laying the groundwork for future enhancements in voice-controlled drone technology.

\section{Preread}
To ensure a robust understanding of the technical aspects required for this project, I reviewed foundational materials, including the DJI user manual and the DJI SDK documentation. 
These resources provided critical insights into drone hardware capabilities, control systems, and the SDK features available for integration. 

In addition to official documentation, I consulted a range of blog posts and technical articles exploring \ac{RTMP} streaming optimization. 
These secondary resources helped shape the approach to minimizing latency and enhancing the real-time responsiveness of drone video feeds. 
This preparatory work informed the design and implementation of the \acp{PoC}, ensuring alignment with the project's goals.
